\documentclass[aspectratio=169]{beamer}



\usepackage{luatexja}
\usepackage[ipaex]{luatexja-preset} % IPAexフォントを自動設定

\usepackage[T1]{fontenc}
\usepackage{amsmath}
\usepackage{amssymb}
\usepackage{listings} % コード表示用
\usepackage{tabularx}


\usetheme{LightTheme}
\setbeamertemplate{footline}[frame number]
\setbeamertemplate{navigation symbols}{}
\setlength{\baselineskip}{10pt}
\setbeamertemplate{caption}[numbered]
\begin{document}

% Listingの設定(XML例の表示に使用)
\lstset{
    language=XML,                 % XMLコードの場合
    basicstyle=\ttfamily\footnotesize,  % ベースフォント
    backgroundcolor=\color{gray!10},    % 背景を薄いグレーに
    frame=single,                 % 枠をつける
    rulecolor=\color{gray!70},    % 枠線の色
    numbers=left,                 % 行番号を左に
    numberstyle=\tiny\color{gray},% 行番号の色
    keywordstyle=\color{blue},    % キーワードの色
    stringstyle=\color{orange!80!black},% 文字列の色
    commentstyle=\color{green!40!black},% コメントの色
    breaklines=true,              % 長い行を折り返す
    showstringspaces=false,       % 空白を記号で表示しない
    tabsize=2,                    % タブ幅
    captionpos=b                  % キャプションを下に
}



\title{How to Design a REST API}
\author{B4 小林紹子}
\date{\today}


% タイトルスライド
\begin{frame}
    \titlepage
\end{frame}

% 目次
\begin{frame}{目次}
    \tableofcontents
\end{frame}

% ----------------------------------------------------
% セクション 1: オブジェクト・モデリング
% ----------------------------------------------------
\section{概要}
\begin{frame}{概要}
  \begin{itemize}
    \setlength{\parskip}{1.5em}
    \item RESTを断片的に学ぶことと,実際にアプリケーション開発に適用することは全く別の課題.
    \item 今回はネットワークベースのアプリケーション向けのREST APIの設計方法を紹介.
    \item アプリケーションの設計プロセスでRESTの原則を適用させる方法に注目.
  \end{itemize}
\end{frame}

\begin{frame}{REST APIの4つの設定ステップ}
    \begin{columns}
    \begin{column}{0.58\textwidth}
      \begin{itemize}
        \setlength{\parskip}{0.5em}
        \item \textbf{Object Modeling}:\\リソースとして提示するオブジェクトの特定.
        \item \textbf{Create Resource URIs}:\\APIのエンドポイントとなるリソースURIの決定.
        \item \textbf{Resource Representations}:\\各URIのリソース表現の設定.
        \item \textbf{Assign HTTP Methods}:\\可能なすべての操作を決定.それらをHTTPメソッドを介してリソースURIにマッピング.
      \end{itemize}
    \end{column}
    \begin{column}{0.48\textwidth}
      \begin{figure}
        \includegraphics[scale=0.35]{StepToDesign.png}
        \caption{\footnotesize{Step to Design a REST API}}
      \end{figure}
    \end{column}
  \end{columns}
\end{frame}
\section{Step1. Object Modeling}
\begin{frame}{Step1:Object Modeling (リソース特定)}
    \begin{itemize}
        \setlength{\parskip}{1em}
        \item \textbf{目的}:\\アプリケーションで扱うオブジェクト(例:データやデバイス)を特定し,\textbf{リソース}として定義.
        \item \textbf{主要リソース例}:
        \begin{itemize}
            \item \textbf{Devices} (デバイス)
            \item \textbf{Configurations} (設定)
        \end{itemize}
        \item \textbf{リソース同士の関係性}:
        \begin{itemize}
            \item ConfigurationはDeviceの\textbf{サブリソース}として機能することも有.
            \item \textbf{例}:あるデバイスに関する設定は,そのデバイスなしには存在しない.
        \end{itemize}
        \item \textbf{識別子}: 全てのリソースは一意の識別子(例: \texttt{id})を持つ.
    \end{itemize}
\end{frame}

% ----------------------------------------------------
% セクション 2: リソースURIの作成
% ----------------------------------------------------
\section{Step2. Create Model URIs}
\begin{frame}[allowframebreaks]{Step2:Create Model URIs(リソースURIの作成)}
    \begin{itemize}
      \setlength{\parskip}{1.5em}
        \item \textbf{目的}:リソースの関係に基づき,URIの構造を決定.
        \item \textbf{設計原則}: URIはリソースを識別する\textbf{名詞}であるべき.
        \item \textbf{禁止事項}: URIに\textbf{操作(動詞)を含めない}.\\(\texttt{/getDevice}や\texttt{/createConfiguration}などは不適切).
    \end{itemize}
    \vspace{0.3em}
    \begin{block}{URI構造の決定}
        \begin{enumerate}
          \setlength{\parskip}{0.5em}
            \item \textbf{トップレベル・コレクション}: 複数形の名詞を使用.
            \begin{itemize}
                \item \texttt{/devices}
                \item \texttt{/configurations}
            \end{itemize}
            \item \textbf{単一リソース}: コレクションにIDを付加.
            \begin{itemize}
                \item \texttt{/devices/\{id\}}
                \item \texttt{/configurations/\{id\}}
            \end{itemize}
            \item \textbf{サブリソース}: 親リソースのパスの下に配置.
            \begin{itemize}
                \item \texttt{/devices/\{id\}/configurations}
                \item \texttt{/devices/\{id\}/configurations/\{configId\}}
            \end{itemize}
        \end{enumerate}
    \end{block}
    \small{※今回は,デバイスをトップレベルリソースとし,設定がデバイスのサブリソースとする.}
\end{frame}


% ----------------------------------------------------
% セクション 3: リソース表現の決定(HATEOAS)
% ----------------------------------------------------
\section{Step3. Determine Resource Representations}
\begin{frame}{Step3. \Large{Determine Resource Representations(リソース表現の決定)}}
  \begin{itemize}
    \setlength{\parskip}{1.5em}
    \item Step2まででリソースURIが決定された.
    \item 今回はそのリソース表現(データ形式)に取り組む.
    \item ほとんどは\textbf{XML},\textbf{JSON}で定義される.
    \item 今回はXMLの例を見ていく.
  \end{itemize}
\end{frame}

\subsection{3.1. Collection Resource}
\begin{frame}[allowframebreaks]{3.1. コレクションリソースの場合(例: \texttt{/devices})}
    \begin{itemize}
        \setlength{\parskip}{2em}
        \item \textbf{内容}: \textbf{最も重要な情報のみ}を含め,完全な詳細は省略.
        \item \textbf{目的}: レスポンスのペイロードサイズを小さく保ち,パフォーマンスを向上.
        \item サブコレクションの場合は,プライマリ・コレクションのサブセットであるため,プライマリ・コレクションと異なるデータフィールドを作成しない.同じ表現フィールドを使用.
    \end{itemize}
        \newpage
        \lstinputlisting[firstline=1, lastline=29, ,language=XML]{sample.xml} % 元テキストのXML例をシミュレート
\end{frame}

\subsection{3.2. Single Resource}
\begin{frame}[allowframebreaks]{3.2. 単一リソースの場合(例: \texttt{/devices/\{id\}})}
    \begin{itemize}
      \setlength{\parskip}{1em}
        \item \textbf{内容}: \textbf{完全な情報}を全て含める.
        \item サブリソースや関連する操作へのリンクリストも含める.
        \item これによりHATEOASにすることが可能.
    \end{itemize}
    \begin{block}{HATEOAS (Hypermedia As The Engine Of Application State)}
        \begin{itemize}
            \item クライアントは,APIの\textbf{次に取り得る操作}を,リソース表現に含まれる\textbf{リンク}を通じて動的に発見できるべき.
        \end{itemize}
    \end{block}
    \newpage
    \lstinputlisting[firstline=31, lastline=58, ,language=XML]{sample.xml} % 元テキストのXML例をシミュレート
    \newpage
    \begin{itemize}
        \item \textbf{単一デバイスリソースの例}:
        \begin{itemize}
          \setlength{\parskip}{1em}
            \item サブリソース(Configuration)へのリンクを含める.
            \item そのリソースに対して実行可能なカスタム操作を\texttt{<method>}タグで示す.
        \end{itemize}
        \lstinputlisting[firstline=51, lastline=58, ,language=XML]{sample.xml} % 元テキストのXML例をシミュレート
    \end{itemize}
\end{frame}




% ----------------------------------------------------
% セクション 4: HTTPメソッドの割り当て
% ----------------------------------------------------
\section{Step4. Assigning HTTP Methods}
\subsection{データの参照(GET)}


\begin{frame}[allowframebreaks]{データの参照(GET)}
  \begin{itemize}
    \setlength{\parskip}{1.5em}
    \item リソースURIとその表現が確定したので,アプリケーションで可能なすべての操作を決定し,それらをリソースURIにマッピングする.\\
    \item 操作の性質(参照、作成、更新、削除)に基づき、適切なHTTPメソッド(\textbf{べき等性}を考慮)を選択.
  \end{itemize}
\end{frame}

\begin{frame}{データの参照(GET)}
データの取得には,すべて\textbf{GET}メソッドを使用(安全かつべき等).

\renewcommand{\arraystretch}{1.5}
\begin{tabularx}{\textwidth}{|l|l|X|}

\hline
\textbf{対象} & \textbf{URIの例} & \textbf{備考} \\
\hline
\hline
\textbf{全コレクション} & \texttt{/devices} & \footnotesize{サイズが大きい場合は,\texttt{?startIndex=\dots}で範囲選択可能.}\\
\hline
\textbf{特定デバイス} & \texttt{/devices/\{id\}} & \textbf{完全な情報}を返す.\\
\hline
\textbf{サブコレクション} & \texttt{/devices/\{id\}/configurations} & 特定のデバイスに紐づく設定リスト.\\
\hline
\end{tabularx}
\vspace{2em}
\lstinputlisting[firstline=60, lastline=62 ,language=XML]{sample.xml} % 元テキストのXML例をシミュレート
\end{frame}

% ----------------------------------------------------------------------

% --- 4.6 ---
\subsection{デバイスまたは設定の更新(PUT)}
\begin{frame}{デバイスまたは設定の更新(PUT)}
  更新操作はべき等な操作であるため,\textbf{PUT} メソッドを使用.

  \lstinputlisting[firstline=63, lastline=64 ,language=XML]{sample.xml} % 元テキストのXML例をシミュレート

PUT メソッドのレスポンス例:
\lstinputlisting[firstline=66, lastline=73 ,language=XML]{sample.xml} % 元テキストのXML例をシミュレート


\end{frame}

% --- 4.7 ---
\subsection{デバイスまたは構成の削除(DELETE)}
\begin{frame}{4.7. デバイスまたは構成の削除}
削除操作は常にDELETEを使用.

\lstinputlisting[firstline=75, lastline=76 ,language=XML]{sample.xml} % 元テキストのXML例をシミュレート

削除が成功した場合のレスポンス:
\begin{itemize}
  \item 非同期削除(キュー登録)→ \texttt{202 Accepted}
  \item 同期削除(即時削除)→ \texttt{200 OK} または \texttt{204 No Content}
\end{itemize}

非同期操作では,成功/失敗を追跡するために \textbf{タスクID} を返す.

\bigskip

※サブリソース削除時は挙動を検討すべき.\\
一般的には「\textbf{ソフトデリート}」を行う.(ステータスを \texttt{INACTIVE} に設定.)\\
これにより他の参照を消す必要がなくなる.


\end{frame}

% --- 4.8 ---
\subsection{設定の適用/削除(PUT/DELETE)}
\begin{frame}{4.8. 設定の適用/削除(PUT/DELETE)}
実際のアプリケーションでは,設定をデバイスに適用または削除する操作が必要.\\
この場合も冪等性を保つために \textbf{PUT} と \textbf{DELETE} を使用.\\
\bigskip
設定をデバイスに適用
\lstinputlisting[firstline=78, lastline=78 ,language=XML]{sample.xml} % 元テキストのXML例をシミュレート

設定をデバイスから削除
\lstinputlisting[firstline=80, lastline=80 ,language=XML]{sample.xml} % 元テキストのXML例をシミュレート

\bigskip
設定そのものを削除するのではなく,デバイスとの関連付けを解除する操作である.

\end{frame}

\begin{frame}{参考文献}
  \begin{itemize}
    \item How To Design a REST API\\url{https://restfulapi.net/rest-api-design-tutorial-with-example/}
  \end{itemize}
\end{frame}
\end{document}