\documentclass[aspectratio=169]{beamer}
\usepackage{amsmath}
\usepackage{amssymb}
\usepackage{amsfonts}
\usepackage{graphicx}
\usepackage{luatexja} 
\usepackage{comment}
\usepackage{bm}
\usepackage{setspace}
\usepackage{caption}
\usepackage{hyperref}
\usepackage{natbib}
\bibliographystyle{unsrt}

\usetheme{LightTheme}
\setbeamertemplate{footline}[frame number]
\setbeamertemplate{navigation symbols}{}
\setlength{\baselineskip}{10pt}

\usepackage{listings}
\usepackage{xcolor}

\lstset{
  basicstyle=\ttfamily\footnotesize,
  keywordstyle=\color{blue}\bfseries,
  stringstyle=\color{orange},
  commentstyle=\color{gray},
  showstringspaces=false,
  frame=single,
  breaklines=true,
  numbers=left,
  numberstyle=\tiny,
}


\begin{document}

% タイトルフレーム
\title{\Large 木幅アルゴリズムの学習システムの構築}
\subtitle{進捗状況} 
\author{\small B4 小林紹子} % 必要に応じて変更・削除
\date{\small\today} % 必要に応じて変更・削除

\begin{frame}
    \titlepage
\end{frame}

\begin{frame}{目次}
    \tableofcontents
\end{frame}


\section{画像生成について}
\begin{frame}{react-graph-visについて}
    \begin{itemize}
        \setlength{\parskip}{1.5em}
        \item 調べた結果,画像を自動で作成するのは難しそう.
        \item 画像を事前に保存してデータベース登録する方法になる.
        \item したがって,React Flowを使用する方向.
        \item うまく図形になるように調整の仕方を調査する.
    \end{itemize}
\end{frame}
\section{データベース使用技術}

\begin{frame}{データベースの使用技術}
    \begin{itemize}
        \setlength{\parskip}{1.5em}
        \item \textbf{Prisma ORM}
        \item \textbf{SQlite}
    \end{itemize}
\end{frame}
\subsection{Prisma}
\begin{frame}{Prisma}
    \begin{itemize}
        \setlength{\parskip}{1.5em}
        \item Node.jsとTypescript向けのオープンソースのORM\cite{prisma}.
        \item ORM(Object Relational Mapping):\\プログラミング言語のエンティティ (オブジェクト) とそれに対応するデータベース要素との関係を抽象化するプロセス\cite{orm}.\\
        \rightarrowfill SQL文を直接書く必要がなく,データベース操作を簡単にしてくれる.
        \item \textbf{特徴}
        \begin{itemize}
            \setlength{\parskip}{1em}
            \item 型安全性:Typescriptとの相性がいいため,コンパイル時にエラーを検出しやすい.
            \item 直感的なAPI:SQLを書く必要がなく,オブジェクトを操作する感覚でデータベース操作が可能.
            \item データベース非依存:MySQL, SQliteなど様々なDBMSに対応.
        \end{itemize}
    \end{itemize}
\end{frame}

\begin{frame}{Prisma ORMの機能}
    \begin{itemize}
        \setlength{\parskip}{1em}
        \item \textbf{Prisma Client}:アプリからDBを操作するためのライブラリ.
        \item \textbf{Prisma Migrate}:データベースの変更記録と,schema.prismaに書いた内容をDBに反映する技術.
        \item \textbf{Prisma Schema}:テーブル構造やデータベースの種類を定義するファイル.
        
    \end{itemize}
    \rightarrow これらをschema.prismaというファイルに書いてDB操作していく.
\end{frame}
\subsection{SQLite}
\begin{frame}{SQLite}
    \begin{itemize}
        \setlength{\parskip}{1.5em}
        \item オープンソースで軽量のRDBMS.
        \item サーバーとしてではなくアプリケーションに組み込むことで利用.
        \item ライブラリとして使用可能なため設定不要な\textbf{自己完結型システム}.
        \item マルチプラットフォームに対応(Linux, Windows, iOSなど).
        \item 1つのテーブルカラムの中に,複数のデータ型を格納可能.
    \end{itemize}    
\end{frame}

\section{データベース構築の流れ}
\begin{frame}[allowframebreaks]{Next.js + React + Prisma + SQLite によるデータベース構築の流れ}
    \begin{enumerate}
        \item \textbf{Prisma のセットアップ}  
        \begin{itemize}
            \item パッケージをインストール  
            \item 初期化コマンドで \texttt{schema.prisma} を生成  
            \item SQLite と接続設定を行う
        \end{itemize}

        \item \textbf{スキーマ定義(DB設計)}  
        \begin{itemize}
            \item \texttt{prisma/schema.prisma} にモデルを記述  
            \item 例:Problem テーブル(id, text, image, answer)
        \end{itemize}

        \item \textbf{マイグレーション実行}  
        \begin{itemize}
            \item \texttt{npx prisma migrate dev --name init}  
            \item スキーマをもとにSQLiteファイル(\texttt{dev.db})を生成
        \end{itemize}

        \item \textbf{Prisma Client 生成}  
        \begin{itemize}
            \item TypeScript からDB操作が可能になる  
            \item \texttt{@prisma/client} を利用
        \end{itemize}

        \item \textbf{Next.js API Routes の作成}  
        \begin{itemize}
            \item \texttt{app/api/problems/route.ts} にAPIを定義  
            \item \texttt{GET}:データ取得,\texttt{POST}:データ登録
        \end{itemize}

        \item \textbf{フロントエンド(React)でデータ表示}  
        \begin{itemize}
            \item \texttt{fetch("/api/problems")} でサーバーから取得  
            \item 問題文や画像を動的に表示
        \end{itemize}
    \end{enumerate}
\end{frame}

\begin{frame}{システム構成図(概要)}
    \begin{center}
        \begin{tikzpicture}[node distance=1.5cm, auto]
            % Nodes
            \node[draw, rounded corners, fill=blue!10, minimum width=3cm, minimum height=1cm] (frontend) {React (Next.js Frontend)};
            \node[draw, rounded corners, fill=green!10, minimum width=3cm, minimum height=1cm, right=of frontend] (api) {Next.js API Route};
            \node[draw, rounded corners, fill=orange!10, minimum width=3cm, minimum height=1cm, right=of api] (db) {SQLite + Prisma};

            % Arrows
            \draw[->, thick] (frontend) -- node[above,yshift=6mm]{fetch()} (api);
            \draw[->, thick] (api) -- node[above,yshift=6mm]{Prisma Query} (db);
        \end{tikzpicture}
    \end{center}
    \vspace{1em}
    \begin{itemize}
        \item フロント(React): UI表示・ユーザ入力
        \item APIルート(Next.js): DB操作用の中間層
        \item Prisma + SQLite: データ永続化
    \end{itemize}
\end{frame}

\begin{frame}{ファイル構造}

app/\\
 ├── api/\\
 │   ├── problems/\\
 │   │   ├── route.ts          ← GET, POST (出題・解答API)\\
 │   └── generate/\\
 │       ├── route.ts          ← 問題をDBに追加するAPI\\
 ├── components/\\
 │   └── ProblemCard.tsx       ← 問題表示用コンポーネント\\
 ├── lib/\\
 │   └── prisma.ts             ← Prismaクライアント\\
 ├── src/\\
      └── page.tsx                  ← 問題ページ (React Flow埋め込み)\\

\end{frame}
\begin{frame}[fragile]{Step 1: Prisma の初期化}
\begin{lstlisting}[language=bash]
# Prisma の導入
npm install prisma --save-dev
npx prisma init

# SQLite を使用するように設定 (.env)
DATABASE_URL="file:./dev.db"
\end{lstlisting}
\end{frame}

%------------------------------------------------------
\begin{frame}[fragile]{Step 2: Prisma スキーマ定義(prisma/schema.prisma)}
\begin{lstlisting}[language=SQL]
datasource db {
  provider = "sqlite"
  url      = env("DATABASE_URL")
}

generator client {
  provider = "prisma-client-js"
}

model Problem {
  id      Int     @id @default(autoincrement())
  text    String
  image   String?
  answer  Boolean
}
\end{lstlisting}
\end{frame}

%------------------------------------------------------
\begin{frame}[fragile]{Step 3: データベース作成}
\begin{lstlisting}[language=bash]
# マイグレーションの作成と反映
npx prisma migrate dev --name init

# データベースを確認
npx prisma studio
\end{lstlisting}
\end{frame}

%------------------------------------------------------
\begin{frame}[fragile]{Step 4: Prisma Client の使用(src/lib/prisma.ts)}
\begin{lstlisting}[language=Java]
import { PrismaClient } from "@prisma/client";
const prisma = new PrismaClient();

export default prisma;

\end{lstlisting}
これにより,どのAPIルートでもDBに接続可能となる.
\end{frame}

%------------------------------------------------------

\begin{frame}[fragile,allowframebreaks]{Step 5: バックエンド側(src/app/api/problems/route.ts)}
\begin{lstlisting}[language=Java]
import { NextResponse } from "next/server";
import prisma from "@/lib/prisma";

// GET: 問題一覧を取得
export async function GET() {
  const problems = await prisma.problem.findMany();
  return NextResponse.json(problems);
}
// POST: 回答を送信し、正誤を判定
export async function POST(req: Request) {
  const { id, answer } = await req.json();
  const problem = await prisma.problem.findUnique({ where: { id } });

  if (!problem) {
    return NextResponse.json({ correct: false, message: "問題が見つかりません" });
  }

  const correct = problem.answer === (answer === "true");
  return NextResponse.json({
    correct,
    message: correct ? "正解!" : "不正解",
  });
}

\end{lstlisting}
\end{frame}

\begin{frame}[fragile,allowframebreaks]{Step 5: フロントエンドで取得(src/app/problems/page.tsx)}
\begin{lstlisting}[language=Java]
"use client";
import { useEffect, useState } from "react";
%型定義
type Problem = {
  id: number;
  text: string;
  image?: string;
};
//DBからデータの取得
export default function ProblemsPage(){
  const [problems, setProblems] = useState<Problem[]>([]);
  const [message,setMessage] = useState("");

  useEffect(() => {
    fetch("/api/problems")
    .then((res) => res.json())
    .then((data) => setProblems(data));
  },[]);
//回答の送信
  async function handleAnswer(id: number,answer: boolean) {
    const res = await fetch("/api/problems",{
      method: "POST",
      headers: {"Context-Type": "application/json"},
      body: JSON.stringify({id,answer}),
    });
    const data = await res.json();
    setMessage(data.message);
    
  }

return(
  <div style={{padding: 20}}>
    <h1>◎× 問題</h1>

    {problems.map((p) => (
      <div key={p.id} style={{marginBottom: 30}}>
        <p>{p.text}</p>
        {p.image && <img src={p.image} alt="問題画像" width={200}></img>}
        <div>
          <button onClick={() => handleAnswer(p.id,true)}>◎</button>
          <button onClick={() => handleAnswer(p.id,false)}></button>
        </div>
  </div>
))}
  {message && <h3>{message}</h3>}
  </div>
);
}
\end{lstlisting}
\end{frame}
\section{今後の課題}
\begin{frame}{今後の課題}
    \begin{itemize}
         \setlength{\parskip}{1.5em}
        \item 正解・不正解をそれぞれ個別に出力・保存.
        \item 画像の出力.
        \item 今は◎×問題だけだが,数値の入力問題でも正答判定が可能.
        \item 問題が一覧でなく,ページ分割可能.
    \end{itemize}
\end{frame}

\section{参考文献}
\begin{frame}[allowframebreaks]{参考文献}
    \small
    \bibliography{refs}    
\end{frame}
\end{document}
