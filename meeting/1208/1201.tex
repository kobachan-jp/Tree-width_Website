\documentclass[aspectratio=169]{beamer}
\usepackage{amsmath}
\usepackage{amssymb}
\usepackage{amsfonts}
\usepackage{graphicx}
\usepackage{luatexja} 
\usepackage{comment}
\usepackage{bm}
\usepackage{setspace}
\usepackage{caption}
\usepackage{hyperref}
\usepackage{natbib}
\usepackage{ulem}

\bibliographystyle{unsrt}

\usetheme{LightTheme}
\setbeamertemplate{footline}[frame number]
\setbeamertemplate{navigation symbols}{}
\setlength{\baselineskip}{10pt}

\usepackage{listings}
\usepackage{xcolor}

\lstset{
  basicstyle=\ttfamily\footnotesize,
  keywordstyle=\color{blue}\bfseries,
  stringstyle=\color{orange},
  commentstyle=\color{gray},
  showstringspaces=false,
  frame=single,
  breaklines=true,
  numbers=left,
  numberstyle=\tiny,
}

\AtBeginSection[]{
  \begin{frame}{目次}
    \tableofcontents[currentsection]
  \end{frame}
}


\begin{document}

% タイトルフレーム
\title{\Large 木幅アルゴリズムの学習システムの構築}
\subtitle{問題表示画面作成}
\author{\small B4 小林紹子} % 必要に応じて変更・削除
\date{\small\today} % 必要に応じて変更・削除

\begin{frame}
	\titlepage
\end{frame}

\begin{frame}{目次}
	\tableofcontents
\end{frame}

\section{概要}
\begin{frame}{前回までの進捗状況と課題}
	\begin{enumerate}
		\setlength{\parskip}{1.5em}
		\item \sout{◎×問題は正答判定可能だが,数値の入力問題でも正答判定が不可能.}
		\item \sout{それぞれの問題の正解・不正解が保存されない.}
		\item \sout{React Flowで思ったようにグラフが描けない.}
		\item 問題文と画像生成が連携されていない.
		\item 問題がページ分割されずに一覧で出てくる.
		\item サーバーをどれにするか調査する.
	\end{enumerate}
\end{frame}


\begin{frame}{今回の進捗状況}
	\begin{enumerate}
		\setlength{\parskip}{1.5em}
		\item ○×問題以外にも対応できる解答欄作成.
		\item 問題文と画像生成が連携.
		\item 問題を一覧でなくページ分割させる.
	\end{enumerate}
	\vspace{1.5em}
	$\rightarrow$\textbf{大幅なデータベース構成の変更とコンポーネント作成を行う.}
\end{frame}

\section{データベースの構成を変更}



\begin{frame}[allowframebreaks]{データベース相関図}
	\begin{figure}
		\includegraphics[scale=0.5]{table1.png}
		\caption{SectionとProblemの関係}
	\end{figure}
	\begin{figure}
		\includegraphics[scale=0.3]{table2.png}
		\caption{個々の問題と画像の関係}
	\end{figure}
\end{frame}

\begin{frame}{問題取得までの流れ}
	\begin{enumerate}
		\setlength{\parskip}{1.5em}
		\item SectionテーブルからProblemテーブルで問題のリストを取得.
		\item ProblemテーブルのcategoryとquestionIdでカテゴリ別のテーブルを参照.\\\vspace{1em}
		      $\implies$category別にAPIでテーブルを選択し,questionIdをそのテーブルの主キーとして参照させる.
		\item それぞれの問題から問題文,答え,グラフなどのデータを表示.\\
		\item グラフや木構造はノードやエッジの情報からReactFlowで描画させる.
	\end{enumerate}
\end{frame}

\begin{frame}{データベース変更により前回までの課題All Clear}
	\begin{enumerate}
		\setlength{\parskip}{1.5em}
		\item Sectionごとにページ分割(Sectionテーブル作成).
		\item 問題文と画像を連携(各カテゴリ別の問題テーブルとGraph&Treeテーブルを外部キーで接続).
		\item 回答欄をカテゴリ別に適したものを表示(カテゴリ別に問題テーブルを作成).
	\end{enumerate}
\end{frame}

\section{現在の問題表示画面作成状況}

\begin{frame}{問題の表示}
	\begin{figure}
		\includegraphics[scale=0.7]{ProblemItem.png}
		\caption{各問題の配置}
	\end{figure}
\end{frame}

\begin{frame}{グラフと木構造の表示}
	\begin{figure}
		\fbox{\includegraphics[scale=0.2]{Graph_Tree.png}}
	\end{figure}
	\center{problems/\$\{number\}がSectionテーブルのidと一致.}
\end{frame}

\begin{frame}{○×問題の場合}
	\begin{columns}
		\begin{column}{0.48\textwidth}
			\begin{figure}
				\fbox{\includegraphics[scale=0.25]{TrueOrFalse.png}}
			\end{figure}
		\end{column}
		\Large$\implies$
		\begin{column}{0.48\textwidth}
			\begin{figure}
				\fbox{\includegraphics[scale=0.19]{TrueOrFalse2.png}}
			\end{figure}
		\end{column}
	\end{columns}
\end{frame}

\begin{frame}{入力問題の場合}
	\begin{columns}
		\begin{column}{0.48\textwidth}
			\begin{figure}
				\fbox{\includegraphics[scale=0.2]{Input.png}}
			\end{figure}
		\end{column}
		\Large$\implies$
		\begin{column}{0.48\textwidth}
			\begin{figure}
				\fbox{\includegraphics[scale=0.2]{Input2.png}}
			\end{figure}
		\end{column}
	\end{columns}
\end{frame}

\begin{frame}{選択問題の場合}
	\begin{columns}
		\begin{column}{0.48\textwidth}
			\begin{figure}
				\fbox{\includegraphics[scale=0.2]{Choice.png}}
			\end{figure}
		\end{column}
		\Large$\implies$
		\begin{column}{0.48\textwidth}
			\begin{figure}
				\fbox{\includegraphics[scale=0.19]{Choice2.png}}
			\end{figure}
		\end{column}
	\end{columns}
\end{frame}

\begin{frame}{ページ遷移}
	\begin{figure}
		\fbox{\includegraphics[scale=0.3]{Prev_Next.png}}
	\end{figure}
	\vspace{2em}
	\center{それぞれSectionが存在する場合のみボタンが登場.}
\end{frame}

\section{今後の予定}

\begin{frame}{今後の予定}
	\begin{itemize}
		\setlength{\parskip}{1.5em}
		\item 問題追加画面作成.
		\item データベースの構成変更.
		\item CSSに凝る.
	\end{itemize}
\end{frame}
\end{document}

