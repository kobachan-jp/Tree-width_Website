\documentclass[aspectratio=169]{beamer}
\usepackage{amsmath}
\usepackage{amssymb}
\usepackage{amsfonts}
\usepackage{graphicx}
\usepackage{luatexja} 
\usepackage{comment}
\usepackage{bm}
\usepackage{setspace}
\usepackage{caption}
\usepackage{hyperref}
\usepackage{natbib}
\usepackage{ulem}

\bibliographystyle{unsrt}

\usetheme{LightTheme}
\setbeamertemplate{footline}[frame number]
\setbeamertemplate{navigation symbols}{}
\setlength{\baselineskip}{10pt}

\usepackage{listings}
\usepackage{xcolor}

\lstset{
  basicstyle=\ttfamily\footnotesize,
  keywordstyle=\color{blue}\bfseries,
  stringstyle=\color{orange},
  commentstyle=\color{gray},
  showstringspaces=false,
  frame=single,
  breaklines=true,
  numbers=left,
  numberstyle=\tiny,
}

\AtBeginSection[]{
  \begin{frame}{目次}
    \tableofcontents[currentsection]
  \end{frame}
}


\begin{document}

% タイトルフレーム
\title{美術館記録アプリ}
\author{ B4-15-26 小林紹子} % 必要に応じて変更・削除
\date{\small\today} % 必要に応じて変更・削除

\begin{frame}
	\titlepage
\end{frame}

\begin{frame}{目次}
	\tableofcontents
\end{frame}

\section{題材の説明}
\begin{frame}{題材の説明}
	\begin{itemize}
		\setlength{\parskip}{0.5em}
		\item 美術館情報を管理・閲覧できるWebアプリ
		\item 主な機能:
		      \begin{itemize}
			      \item 美術館・博物館の一覧表示
			      \item 展覧会の一覧表示
			      \item 詳細ページで情報確認
			      \item 登録・編集・削除
		      \end{itemize}
		\item データベース:SQLite
		\item 使用技術:
		      \begin{itemize}
			      \item フレームワーク:Next.js
			      \item ORM:Prisma 
		      \end{itemize}
	\end{itemize}
\end{frame}

\section{CRUD操作}
\begin{frame}[allowframebreaks]{REST API の設計}
	\small{Next.jsのApp Routerの規則よりファイル構造がそのままパスになるため/api/rest/ができてしまった.}
	\begin{itemize}
		\setlength{\parskip}{1em}
		\item GETリクエスト
		      \begin{itemize}
			      \item /api/rest/museums $\implies$ 美術館一覧取得
			      \item /api/rest/museums/:id $\implies$ 該当する美術館の情報取得
			      \item /api/rest/exhibitions $\implies$ 展覧会一覧取得
			      \item /api/rest/exhibitions/:id $\implies$ 該当する展覧会の情報取得
		      \end{itemize}

		\item POSTリクエスト
		      \begin{itemize}
			      \item /api/rest/museums $\implies$ 美術館を登録
			      \item /api/rest/exhibitions $\implies$ 展覧会を登録
		      \end{itemize}
		\newpage					
		\item PUTリクエスト
		      \begin{itemize}
			      \item /api/rest/museums/:id $\implies$ 該当する美術館の情報更新
			      \item /api/rest/exhibitions/:id $\implies$ 該当する展覧会の情報更新
		      \end{itemize}

		\item DELETEリクエスト
		      \begin{itemize}
			      \item /api/rest/museums/:id $\implies$ 該当する美術館の情報削除
			      \item /api/rest/exhibitions/:id $\implies$ 該当する展覧会の情報削除
		      \end{itemize}

	\end{itemize}
\end{frame}

\begin{frame}[allowframebreaks]{TOP画面}
	\begin{figure}
		\fbox{\includegraphics[scale=0.25]{top.png}}
	\end{figure}
	\begin{itemize}
		\setlength{\parskip}{1em}
		\item 最新登録展覧会5件を表示.
		\item RESTとGraphQLでは取得した展覧会数が異なる.
		\item また,取得したデータベースのフィールド数も異なる.
		\item その他のページでもフィールド数の比較が一番下に出るようになっている.
	\end{itemize}
\end{frame}

\begin{frame}{気に入っている部分}
	\begin{figure}
		\fbox{\includegraphics[scale=0.15]{map.png}}
	\end{figure}
	美術館詳細ページでは地図の出力とオフィシャルサイトへの移動ができるようにした.\\
\end{frame}


\section{GraphQL Schemaの設計}
\begin{frame}{DB設計}
	\begin{columns}
		\begin{column}{0.48\textwidth}
			\begin{figure}
		\includegraphics[scale=0.32]{museumModel.png}
		\caption{Museumテーブル}
	\end{figure}		
		\end{column}
		\begin{column}{0.48\textwidth}
			\begin{figure}
		\includegraphics[scale=0.32]{exhibitionModel.png}
		\caption{Exhibitionテーブル}
	\end{figure}		
		\end{column}
	\end{columns}
	\center{Museum(1)$\iff$ Exhibition(多)}
\end{frame}

\begin{frame}{GraphQL設計(Schema)}
	\begin{figure}
		\includegraphics[scale=0.4]{graphql-schema.png}
		\caption{Schema設計}
	\end{figure}
	
\end{frame}

\begin{frame}{各クエリ(一部)}
	エンドポイント:'/api/graphql'
	\begin{columns}
		\begin{column}{0.3\textwidth}
			\begin{figure}
				\includegraphics[scale=0.5]{app-query.png}
				\caption{最新登録展覧会}
			\end{figure}
		\end{column}
		\begin{column}{0.25\textwidth}
			\begin{figure}
				\includegraphics[scale=0.5]{exhibitions-query.png}
				\caption{展覧会一覧}
			\end{figure}
		\end{column}
		\begin{column}{0.35\textwidth}
			\begin{figure}
				\includegraphics[scale=0.5]{exhibitionId-query.png}
				\caption{展覧会詳細}
			\end{figure}
		\end{column}
	\end{columns}
\end{frame}

\section{RESTとGraphQLの比較・検討・考察}

\begin{frame}[allowframebreaks]{アンダーフェッチ(TOPページにおいて)}
	\begin{figure}
		\fbox{\includegraphics[scale=0.6]{toppage.png}}
	\end{figure}
	展覧会の開催場所がほしい.
		\begin{figure}
			\fbox{\includegraphics[scale=0.6]{underfetch.png}}
			\caption{RESTの場合}
		\end{figure}
		$\implies$ museumIdまでは得られるがname(美術館名)までは取得できない.\\
		$\implies$ nameを得るにはGET(/api/rest/museums/4)などする必要あり.\\
		$\implies$ {\color{red}N+1問題の発生}\\
		
		\begin{columns}
			\begin{column}{0.3\textwidth}
				\begin{figure}
					\includegraphics[scale=0.3]{app-query.png}
					\caption{クエリ}
				\end{figure}
			\end{column}
			\begin{column}{0.7\textwidth}
				\begin{figure}
					\fbox{\includegraphics[scale=0.6]{underfetch-graqhql.png}}
					\caption{GraphQLの場合}
				\end{figure}
			\end{column}
		\end{columns}
		$\implies$ アンダーフェッチの解消
\end{frame}

\begin{frame}[allowframebreaks]{オーバーフェッチ(展覧会の一覧表示)}
	\begin{figure}
		\fbox{\includegraphics[scale=0.5]{exhibitions.png}}
		\caption{展覧会一覧表示}
	\end{figure}
	必要なもの:Exhibition.idとExhibition.titleのみ\\
	\newpage
	\begin{figure}
		\fbox{\includegraphics[scale=0.6]{rest-exhibitions.png}}
		\caption{RESTの場合}
	\end{figure}
	$\implies$ オーバフェッチの発生

	\begin{figure}
		\fbox{\includegraphics[scale=0.6]{graphql-exbitions.png}}
		\caption{GraphQLの場合}
	\end{figure}
	$\implies$ オーバフェッチの解消
\end{frame}

\begin{frame}{RESTとGraphQLの比較(展覧会一覧表示画面)}
    \centering
    \begin{table}
        \begin{tabular}{lcc}
            \hline
            \textbf{比較項目} & \textbf{REST} & \textbf{GraphQL} \\ \hline
            呼び出し回数      & 3            & 1              \\
            レスポンス量      & 7.97KB               & 1.10KB              \\
						サイズ						& 12.01KB									&	739Byte						\\\hline
        \end{tabular}
        \caption{システム変更前後のパフォーマンス比較}
    \end{table}
		$\implies$ 他のページでも一覧取得では同じような差がでた.
\end{frame}

\begin{frame}{考察:RESTとGraphQLの比較}
  \begin{itemize}
		\setlength{\parskip}{1em}
    \item \textbf{小規模構成における有意差}
      \begin{itemize}
        \item わずか2テーブルの構成でも,回数・通信量ともに顕著な差を確認.
        \item これは、GraphQLのクエリ最適化が最小構成から有効であることを示す.
      \end{itemize}
    \item \textbf{実運用の場合}
      \begin{itemize}
        \item テーブル数やリレーションが増加する実環境では,この差はさらに顕著になると推測される.
      \end{itemize}
    \item \textbf{結論}
      \begin{itemize}
        \item 特定フィールドの抽出や複数リソースの統合において,GraphQLは極めて高い通信効率を持つ.
      \end{itemize}
  \end{itemize}
\end{frame}

\section{RESTとGraphQL の適性}
\begin{frame}{本アプリにおける技術選定}
  \begin{block}{結論:GraphQLの採用が適している}
    美術館アプリのような複雑なデータ関連性を持つシステムでは,\textbf{GraphQLの採用が通信効率および開発効率の両面で優位}.
  \end{block}

  \begin{itemize}
		\setlength{\parskip}{1em}
    \item \textbf{リソースの統合:} \\将来的に作品詳細画面で「作者」や「関連作品」を一度に取得できるため,REST特有の「何度もリクエストを送る手間」を排除.
    \item \textbf{通信量の最適化:} \\必要なフィールドを絞り込めるGraphQLは,ユーザーの通信制限や端末負荷を軽減.
    \item \textbf{将来性:} \\今後テーブル数や機能が増加しても,今回の実験結果が示す通り,RESTに比べてパフォーマンスの悪化を最小限に抑制.
  \end{itemize}
\end{frame}

\end{document}


